\documentclass{article}
\usepackage{hyperref}

\begin{document}

\title{Introduction to Haggo}
\author{Author'(s) LowkeyCoding, Anons}

\maketitle

\section{Introduction}
What, where trying to achive is to recreate the old classic Haboo. Habbo (previously known as Habbo Hotel) is a social networking service and online community aimed at teenagers. The website is owned and operated by Sulake, a Finnish corporation. The service began in 2000 and has expanded to include nine online communities (or "hotels"), with users in over 150 countries(\href{https://boards.4channel.org/g/thread/70288715#p70288715}{\textgreater\textgreater 70288715}).

\subsection{Point of whitepaper}
This "Whitepaper" is trying to specify a path for the project to go. The paper will be written, partially complete to leave some decesions up to the /g/ community via poles.

\section{Frontend}
I see two major paths this project can take \href{https://boards.4channel.org/g/thread/70288715#p70288715}{\textgreater \textgreater 70289125} and others are intrested in creating a standalone application. We could also stay more true to the original game and develop it as a hmtl5 game. A example is https://github.com/TheNamesRay/Habbo-in-HTML5 a old abbandoned project.
    \href{https://www.strawpoll.me/17681082}{Frontend pole}
 \section{Backend}
The backend is more open we could go either with nodejs, go, C and many other lanuages but, it seems the community leans more towards C and C++.
    \href{https://www.strawpoll.me/17681079}{Backend pole}

\end{document}
